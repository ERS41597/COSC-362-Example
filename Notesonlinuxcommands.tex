\documentclass [12pt] {article}
\begin {document}
When Logging ito the machine:
\begin{verbatim}
ssh <username>@<machine>
Here are some notes on Linux Commands:
\begin {verbatim}

\end {verbatim}


* cd - Change Directory

* Ls - View the Contents in the Directory

* rm <Name of text file> - Remove a file from a directory

* Nano - Creates a file in the directory youre in.

* Cp - Copy a file from place to a destination

* rf - recursive force.

* Less - File Viewer that shows less...

* Cat - Dumps contents of a stated fle to the terminal.

* Head - shows the start of the file.

* Less - Shows the end of the file.

* Tab - AutoCompletes options on Linux

* history - This shows the last so many commands entered into the terminal

* > - Direct the terminal buffer into specified file.

* >> Appends to the stated file the information in the terminal's buffer.

* echo - This will put text back into the buffer.

* & - This flag after a linux command makes the job run in the background
      Allows the user to keep adding more commands in the terminal.

* firefox - opens a web browser


* kill <process number> - ends the process

\end {verbatim}
\section { the chmod command}

This allows us to change how and who can mess with a file.
File permissions come in three sets:

-(USER)(Group)(Other) with three parts to each
-rwx

For chmod we set the permissions with a number
\begin{itemized}
\item Executehas a tag value 1
\item Wirite ahs a tag calue 2
\item REad ahs a tag value 4
\end{itemized}
These are addiitve. And we use them in a tuple after the chmod command
Before the file we are changing.

Example:

chmod 777 myfile % This gives all access to everyone.

\end{document}
